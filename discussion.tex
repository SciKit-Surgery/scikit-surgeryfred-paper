\section{Discussion}
SciKit-SurgeryFRED has proven useful for the teaching of fiducial based registration,
though more work on usability and measurement of teaching outcomes would be helpful. 
The results of the simulated ablation study 
are also of interest, however more participants are needed before any firm conclusions
can be drawn.
If we assume that additional data follow the same pattern as observed to date we
would need to run the game on a further 15 subjects to show statistical significance
on the difference in scores when using actual \gls{TRE}. In the case of the
\gls{FLE} we would need a further 30 participants. Given the ease with
which \fred can be deployed and used we hope to achieve this in the near future.

Currently \fred implements numerical measures for registration errors, however
in practice it may be better to use
graphical representations of registration error. An example is the use of
outline rendering to show the misalignment of anatomical edges during key hole 
surgery\cite{PMID:29663273}. Azimi et al. \cite{10.1007/978-3-030-59716-0_7} 
provide a similar example of a neuro-surgical guidance system where the 
registration error is communicated to the surgeon via misalignment of 
external anatomy. Such a method could be relatively easily implemented in \fredns, with
gamification used to measure its effectiveness. 

Currently \fred is limited to rigid registration using point correspondence, however, 
there is no architectural reason why it could not be used to examine the 
effects of registration uncertainty more generally. It may be useful to implement 
more general error prediction models for rigid registrations\cite{4359072,5629373}.

Surface based registration has long been proposed as a way to improve 
registration accuracy for neurosurgery \cite{736031} and is 
is now integrated into commercial systems with mixed results \cite{mongen2020}.
\fred could in 
theory be extended include different registration methods, though obviously would 
require a more inclusive name. As we move to non rigid and 
probabilistic registrations, the correct interpretation of registration 
uncertainty will become more challenging \cite{10.1007/978-3-030-59716-0_26}.

\section{Conclusion}
Understanding registration uncertainty is essential for the correct 
use of image guided surgery. However there is a lack of tools to 
aid teaching and research.
We have presented \fred and demonstrated its use for teaching and research. 
\fredns's ease of use enables it be deployed and demonstrated rapidly. However as
it is entirely open source and readily available \fred can also be applied to 
specific user's research questions. 




%Weaknesses:
%Useability study needs more samples, plus it should include a Null case (no information.

%Future work;
%We will include the FLE in the game log file, so we can examine what happens to performance under different conditions of FLE.

%Use FRED to examine other registration methods, for example a skin surface based method as used by medtronic.

%Use FRED to examine other ways of communicating registration error, for example raphical methods like (several, but include Thompson 2018)
