\section{Introduction}
The fact that \gls{FRE} is uncorrelated with \gls{TRE} is well established 
\cite{fitzpatrick2009}. In spite of this, many students and users of clinical guidance systems struggle to 
correctly interpret \gls{FRE} and residual errors in general. 
\fred (Fiducial Registration Educational Demonstrator)
 \cite{stephen_thompson_2020_4314971} was developed using the 
\sksurgery \cite{PMID:32436132} libraries to probe the causes of misconceptions 
about \gls{FRE} and \gls{TRE}. \fred supports online learning and provides tools to enable 
research into user interface design for image guidance systems. 

Faced with the need to deliver teaching and research remotely during 2020's {SARS-CoV-2}\cite{PMID:32123347}
travel restrictions, we 
developed a new online tutorial\footnote{\href{https://mphy0026.readthedocs.io/en/latest/summerschool/registration_demo.html}{https://mphy0026.readthedocs.io/en/latest/summerschool/registration{\textunderscore}demo.html}}
on fiducial based registration. The tutorial utilises \fred which is now implemented as a browser based web-app. 
\fred can be accessed on-line\footnote{\href{https://scikit-surgeryfred.ew.r.appspot.com/}{https://scikit-surgeryfred.ew.r.appspot.com/}} or deployed locally.
Rather than simply telling the students that \gls{FRE} is uncorrelated with \gls{TRE} 
we decided it would be more effective to use divergent questioning strategies\cite{Tofade155} within the application, analysis, synthesis and 
evaluation levels of Bloom's taxonomy \cite{blooms_tax}: hence the question posed within the title of this paper. 

We also took the opportunity to develop a serious game based on \fredns. 
Serious games are games designed to be engaging to play, whilst setting out to achieve an objective
beyond pure entertainment. In the case of \fredns, the objective is to gather data on how clinical 
decision making might be influenced by the communication of registration errors. 
Serious games have significant potential 
as a training tool for clinicians \cite{PMID:28133947, serious-needle}. Serious games have also been shown to be 
useful in assessing the usability of human computer interfaces \cite{hci_games}, which is of particular
relevance to computer aided surgery. The results of the game indicate that the choice of error statistic shown can affect clinical decision making.  

In this paper we introduce the \fred application and show how it can
be used as an educational tool and to perform studies to measure registration and 
ablation performance. By default \fred
implements \gls{FLE} as an isotropic, normally distributed, independent variable. 
In this paper we show how \fred can be modified to implement anisotropic and 
systematic errors, and present the results of these modifications. 
It is our plan to build on 
the software to perform larger studies; to investigate more realistic models of \gls{FLE};
to investigate other registration approaches; and other ways of communicating registration error. 
\fred is entirely open 
source software and we encourage researchers and educators to use and contribute to it. 

