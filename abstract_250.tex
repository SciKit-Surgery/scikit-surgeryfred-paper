Understanding the relationship between fiducial registration error (FRE) and target registration error (TRE) is important for the correct use of interventional guidance systems. Whilst it is well established that TRE is statistically independent of TRE, system users still struggle against the intuitive assumption that a low FRE indicates a low TRE. We present the SciKit-Surgery Fiducial Registration Educational Demonstrator and describe its use. SciKit-SurgeryFRED was developed to enable remote teaching of key concepts in image registration. SciKit-SUrgeryFRED also supports research into user interface design for image registration systems. 

SciKit-SurgeryFRED can be used to enable a remote tutorials covering the statistics relevant to image guided interventions. Students are able to place fiducial markers on pre and intra-operative images and observe the effects of changes in marker geometry, marker count, and fiducial localisation error on TRE and FRE. SciKit-SurgeryFRED also calculates statistical measures for the expected values of TRE and FRE. Because many registrations can be performed quickly the students can then explore potential correlations between the different statistics. 

SciKit-SurgeryFRED also implements a registration based game, where participants are rewarded for complete treatment of a clinical target, whilst minimising the treatment margin. We used this game to perform a remote usability study, measuring how user performance changes depending on what error statistics are made available. The results support the assumption that knowing the exact value of target registration error leads to better treatment. Display of other statistics had little impact on the treatment performance.  
