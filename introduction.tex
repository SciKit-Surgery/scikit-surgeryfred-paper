\section{Introduction}
The fact that \gls{FRE} is uncorrelated with \gls{TRE} is well established 
\cite{fitzpatrick2009}. In spite of this, many students and users of clinical guidance systems struggle to 
correctly interpret \gls{FRE} and residual errors in general. 
\fred (Fiducial Registration Educational Demonstrator)
 \cite{stephen_thompson_2020_4314971} was developed using the 
\sksurgery \cite{PMID:32436132} libraries to probe the causes of misconceptions 
about \gls{FRE} and \gls{TRE}. SciKit-SurgeryFRED supports on line learning and provides tools to enable 
research into user interface design for image guidance systems. 

Faced with the need to deliver teaching and research remotely during 2020's {COVID} travel restrictions, we 
developed a new on line tutorial\footnote{\url{https://mphy0026.readthedocs.io/en/latest/summerschool/registration_demo.html}}
on fiducial based registration. To run the tutorial the student installs SciKit-SurgeryFRED, which is a 
set of three python based applications. Rather than simply telling the students that \gls{FRE} is uncorrelated with \gls{TRE} 
we decided it would be more effective to use divergent questioning strategies\cite{Tofade155} within the application, analysis, synthesis and 
evaluation levels of Bloom's taxonomy \cite{blooms_tax}. Hence the question posed within the title of this paper. 

We also took the opportunity to develop a serious game based on SciKit-SurgeryFRED. Serious games have significant potential 
as a training tool for clinicians \cite{PMID:28133947, serious-needle} and have the potential to enable the collection of substantial amounts of 
usability data. 

In this paper we introduce the SciKit-SurgeryFRED application and show how it can
be used as an educational tool and to perform a usability study. SciKit-SurgeryFRED 
currently implements \gls{FLE} as an isotropic, normally distributed independent variable. 
It is our plan to build on 
the software to perform larger studies; to investigate more realistic models of \gls{FLE};
to investigate other registration approaches; and other ways of communicating registration error. 
SciKit-SurgeryFRED is open 
source software and we encourage researchers and educators to use and contribute to it. 

