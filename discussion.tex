\section{Discussion}
\subsection{Future Work}
Future work will expand the range of \gls{FLE} models implemented, to include anisotropic \cite{4359072} 
errors and systematic errors \cite{6294449}. 

Early work suggests that the choice of numerical information displayed can impact 
clinical decision making. However in practice it may be better to implement 
graphical representations of registration error as we have used for 
liver surgery\cite{PMID:29663273}. Azimi et al. \cite{10.1007/978-3-030-59716-0_7} 
provide a recent example of a neuro-surgical guidance system where the 
registration error is communicated to the surgeon via misalignment of 
external anatomy. Such a method could be relatively easily implemented in \fred, with
gamification used to measure its effectiveness. 

Currently \fred is limited to rigid registration using point correspondence, however, 
there is no architectural reason why it could not be used to examine the 
effects of registration uncertainty more generally. It may be useful to implement 
more general error prediction models for rigid registrations\cite{4359072,5629373}.

As we move to non rigid and 
probabilistic registrations, the correct interpretation of registration 
uncertainty will become more challenging \cite{10.1007/978-3-030-59716-0_26}.

SciKit-SurgeryFRED has proven useful for the teaching of fiducial based registration. The results of the usability study 
are also of interest, however significantly more participants are needed before any firm conclusions can be drawn. This 
should be possible as a major strength of SciKit-SurgeryFRED is that it can be deployed anywhere. 





%Weaknesses:
%Useability study needs more samples, plus it should include a Null case (no information.

%Future work;
%We will include the FLE in the game log file, so we can examine what happens to performance under different conditions of FLE.

%Use FRED to examine other registration methods, for example a skin surface based method as used by medtronic.

%Use FRED to examine other ways of communicating registration error, for example raphical methods like (several, but include Thompson 2018)
